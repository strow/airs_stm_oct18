\documentclass[10pt,t]{beamer}

\usepackage[utf8]{inputenc}
\usepackage[T1]{fontenc}
\usepackage{graphicx}
\usepackage{grffile}
\usepackage{longtable}
\usepackage{wrapfig}
\usepackage{rotating}
\usepackage[normalem]{ulem}
\usepackage{amsmath}
\usepackage{textcomp}
\usepackage{amssymb}
\usepackage{capt-of}
\usepackage{hyperref}

%\newcommand{\nitrous}{N$_2$O\xspace}
%\newcommand{\ozone}{O$_3$\xspace}

\input beamer_setup

\usetheme{default}
% ---------------------------------------------------------------------
% ---------------------------------------------------------------------
\begin{document}

\title[]{15 years of Longwave Flux Trends : \newline
  Roles of CO2, WV, Temperature and clouds, \newline
  using ERA and AIRS data}
\author{Sergio DeSouza-Machado, Larrabee Strow}
\institute{Department of Physics, JCET\\
  University of Maryland Baltimore County (UMBC)}
\date{October 2, 2018}
% ---------------------------------------------------------------------
% ---------------------------------------------------------------------
\begin{frame}
  \titlepage
\end{frame}
% ---------------------------------------------------------------------
\begin{frame}
  \frametitle{Overview}
  \begin{itemize}
  \item We are starting to produce long-term trends directly from gridded (time and space) radiances.
  \item Using an OLR fast model (AER's RRTM) we can compute OLR trends from our retrieved T/Q/cloud, etc trends, and partially validate our observed trends.
  \item This minimizes bias errors in RRTM (and in our cloud trend assumptions).
    potentially validate our trends by comparing to CERES OLR trends
  \item This process is very fast and easy to test in many ways.
    \item Details on our retrieved thermodynamic and cloud rates (3.20 pm talk by Larrabee Strow)
  \item Recent interest in Antarctic Fluxes "Unmasking the negative greenhouse effect over the Antarctic Plateau" by Seijas, Taylor, Cai, \emph{Nature 2018} prompted this work as well.
  \end{itemize}
\end{frame}
% % ---------------------------------------------------------------------
% \section{General idea}
% % ---------------------------------------------------------------------
% \begin{frame}
%   \frametitle{Background}
%   \begin{itemize}
%   \item If there was no atmosphere, then surface heated by sun (or eg nuclear energy within)
%   \item Power radiated out to space would simply be $P/A = OLR = \sigma T_{surf}^4$ 
%   \item However there are clouds and atmospheric gases, transporting heat and moisture globally
%   \item \textcolor{blue}{\cd 15 \um band (Planck peak), BT $\sim$ 250 K or less} \newline
%         \textcolor{blue}{\water Far-IR  BT is also pretty low} \newline
%         \textcolor{red}{Outgoing Longwave Radiation $\ll$ radiated by surface}  
%   \item Rough numbers
%     \begin{small}
%     \begin{itemize}
%     \item Tropical atmosphere $T_{surf}$ = 299.7 K, BT1231 = 296.3 K
%     \item $<T_{surf}> = 285 K, <T_{UA}> = 255 K$ greenhouse effect
%     \item  Incoming TOA solar = 1360 W/m2; cross section = 340 W/m2; albedo = 0.3;
%         so nominally need 238 W/m2 OLR
%     \end{itemize}
%     \end{small}    
%   \end{itemize}
% \end{frame}
% ---------------------------------------------------------------------

%\begin{frame}
%  \frametitle{kCARTA TOA radiances (TRP profile)}

%  \begin{center}
%    \noindent\includegraphics[width=0.625\textwidth]{Figs/generic_planckTOArads.pdf}
%  \end{center}
  
%  \begin{itemize}  
%  \item Red hashlines : kCARTA band edges
%  \item FarIR is a mix of H2004, H2008, H2012, MT-CKD 1
%  \item Cannot really do Far IR cloud calcs
%  \item so use RRTM (black hashlines : RRTM band edges)
%  \end{itemize}

%\end{frame}
% % ---------------------------------------------------------------------
% \begin{frame}
%   \frametitle{kCARTA TOA B(T) Spectrum}

%   \begin{center}
%     \noindent\includegraphics[width=0.75\textwidth]{Figs/generic_planckTOABT.pdf}
%   \end{center}

%   \begin{itemize}  
%   \item Red hashlines : kCARTA band edges
%   \item kCARTA liens (older spectroscopy, no cloud params in Far IR)
%   \item So use RRTM (black hashlines : RRTM band edges)
%   \end{itemize}

% \end{frame}
% ---------------------------------------------------------------------
\begin{frame}
  \frametitle{Total TOA Radiances and Fluxes}

  \hspace{0.50in} TOA BT1231 \hspace{1.75in} Fluxes \\
  \begin{center}
    \dlandgraph{0.48}{Figs/generic_BT1231_vs_lat.pdf}{Figs/generic_olr_vs_lat.pdf}
  \end{center}

  \begin{itemize}
  \item \textcolor{red}{RRTM : Using TwoSlab clouds, quite approximate.}
  \item \begin{small}$r(\nu) = f_{ice} r_{ice}(\nu) + f_{water} r_{water}(\nu) +
    f_{overlap} r_{overlap}(\nu) + f_{clear} r_{clear}(\nu)$ \end{small}
  \item Computed flux has a 10 W/m2 bias relative to CERES, mostly latitude independent
  \end{itemize}
\end{frame}
% ---------------------------------------------------------------------
% ---------------------------------------------------------------------
%\section{Obs vs Calc}
% ---------------------------------------------------------------------
% ---------------------------------------------------------------------
% \begin{frame}
%   \frametitle{AIRS Trends for Window Channel (1231 cm-1)}

%   Earlier Larrabee showed BT1231 Sarta TwoSlab calcs are increasing by
%   0.15 K/yr in the tropics, or $\sim$ 2.25 K over 15 years \newline

%   \begin{columns}
%     \begin{column}{0.65\columnwidth}
%       \begin{center}
%         \noindent\includegraphics[width=\linewidth]{airs15year_lat_trends_900cm-1.png}
%       \end{center}
%     \end{column}

%     \begin{column}{0.35\columnwidth}

%       \vspace{0.1in}
      
%       $dP = \sigma 4 T^3 dT$, with $T \sim 285 K, dT \sim 2.25 K $ $\rightarrow $
%       \textcolor{red}{dP $\sim$ 11 W/m2} \newline

%       That is not observed! \newline
      
%       \vspace{0.1in}
      
%       Can we understand where this comes from?

%     \end{column}    
%   \end{columns}
  
% \end{frame}
% % ---------------------------------------------------------------------
% \begin{frame}
%   \frametitle{AIRS Obs vs ERA changes : OLR}

%   Take averaged profile over Year 01, all latbins, find OLR \newline
%   Take averaged profile over Year 15, all latbins, find OLR \newline
%   Turn on/off clouds, thermo, cloud, and CO2 changes
  
%   \begin{center}
%     \noindent\includegraphics[width=0.625\textwidth]{Figs/deltaolr_vs_lat.pdf}
%   \end{center}
% \end{frame}
% % ---------------------------------------------------------------------
% % ---------------------------------------------------------------------
% \section{Flux Jacobians}
% % ---------------------------------------------------------------------
% % ---------------------------------------------------------------------
\begin{frame}
  \frametitle{Flux Changes : Method}
  \begin{itemize}
  \item Start with one averaged profile per zonal latbin, 370 ppm \cd, 1800 ppb \methane, 320 ppb \nitrous
  \item Now for next 15 years (180 months between 2002/09 to 2017/08)
  \begin{itemize}
    \item include  UMBC T/ST/Q/O3/cloud trends derived from radiance trends.
    \item Add in CO2 change $\cd(t) = 370 + 2.2/12 \delta t$ where $\delta t = (t-2002/08)$ in months, same for CH4, etc.
  \end{itemize} 
  \item Compute RRTM Clear sky and TwoSlab fluxes for all latitude bins/180 months
  \item Can take the differences to get $\delta$ OLR
  \end{itemize}
\end{frame}
% % ---------------------------------------------------------------------
% \begin{frame}
%   \frametitle{ERA Clear Sky Flux Jacobians: Total 15 Years}
%   \vspace{-0.1in}
%   \begin{center}
%     \noindent\includegraphics[width=0.75\linewidth]{clrskly_temp_gas_fluxjacs_lls.pdf}
%   \end{center}
%   \vspace{-0.1in}
%   \begin{itemize}
%   \item Lowered flux by \cd and \water
%   \item Increase flux by Tsurf and T(z), \ozone small change
%   \item Net change is variable except large in N. Polar regions
%   \end{itemize}
% \end{frame}
% % ---------------------------------------------------------------------
% \begin{frame}
%   \frametitle{ERA All Sky Flux Jacobians : Total 15 Years}
%   \vspace{-0.1in}  
%   \begin{center}
%     \noindent\includegraphics[width=0.75\linewidth]{Figs//allsky_cloud_fluxjacs.pdf}
%   \end{center}
%   \vspace{-0.1in}
%   \begin{itemize}
%   \item (ERA) OLR tropical decrease due to ice cloud fraction!
%   \item Not shown : flux jacobians for water clouds (typically much smaller than cirrus flux jacs)
%   \end{itemize}
  
  
% \end{frame}
% % ---------------------------------------------------------------------
% % ---------------------------------------------------------------------
% \section{Flux Changes Using UMBC Spectral Rates Retrievals}
% % ---------------------------------------------------------------------
% % ---------------------------------------------------------------------

\begin{frame}
  \frametitle{\textcolor{red}{\bf Geophysical/cloud rates from L1b spectral rates}}
  \vspace{-0.1in}
  \hspace{0.50in} WV percent/yr \hspace{1.0in} Temperature K/decade \\
  \vspace{-0.1in}  
  \begin{center}
    \dlandgraph{0.425}{Figs//umbc_wv_rates_from_spectralrates.png}{Figs//umbc_T_rates_from_spectralrates.png}
  \end{center}

  \vspace{-0.1in}
  \hspace{0.50in} O3 percent/yr \hspace{1.0in} ST/Cloud trends \\
  \vspace{-0.1in}  
  \begin{center}
    \dlandgraph{0.375}{Figs//umbc_o3_rates_from_spectralrates.png}{Figs//umbc_STnCLD_rates_from_spectralrates.pdf}
  \end{center}

\end{frame}

\begin{frame}
  \frametitle{\textcolor{red}{\bf Comparison to CERES L3 rates}}
  RRTM OLR differences using UMBC trend retrievals.
  \begin{center}
    \noindent\includegraphics[width=0.7\linewidth]{Figs/umbc_vs_ceres_fluxrates.pdf}
  \end{center}
 Errorbars : Put in 5\% uncertainity for CO2 rates, and errorbars for T/WV/O3/ST/clouds from Larrabee; will improve this
\end{frame}

\begin{frame}
  \frametitle{Flux Changes by RRTM bands: 15 Years}
  RRTM allows us to break down trends by wavenumber.\newline 
  \begin{center}
    \noindent\includegraphics[width=0.625\linewidth]{Figs/umbc_vs_band_fluxrates.png}
  \end{center}
\end{frame}

\begin{frame}
  \frametitle{Breakdown of OLR Differences by Cause: A}
%  \begin{center}
%    \noindent\includegraphics[width=0.625\linewidth]{Figs/umbc_total_vs_atm_vs_GHG.png}
%  \end{center}

  \hspace{0.50in} WV  \hspace{1.5in} CO2 \\
  \begin{center}
    \dlandgraph{0.48}{Figs/umbc_wvonly_spectralfluxchange.png}{Figs/umbc_co2only_spectralfluxchange.png}
  \end{center}
\end{frame}

\begin{frame}
  \frametitle{Breakdown of OLR Differences by Cause: B}
%  \begin{center}
%    \noindent\includegraphics[width=0.625\linewidth]{Figs/umbc_total_vs_atm_vs_GHG.png}
%  \end{center}
  \hspace{0.50in} Surf. Temp  \hspace{1.5in} Atm. Temp \\
  \begin{center}
    \dlandgraph{0.48}{Figs/umbc_STonly_spectralfluxchange.png}{Figs/umbc_Tonly_spectralfluxchange.png}
  \end{center}
\end{frame}
% % ---------------------------------------------------------------------
% % ---------------------------------------------------------------------
% \section{Conclusions}
% % ---------------------------------------------------------------------
% % ---------------------------------------------------------------------

\begin{frame}
  \frametitle{Breakdown of OLR Differences by Latitude}
  Integrate the previous results over spectral band to get OLR change versus latitude
  
  % \begin{center}
%    \noindent\includegraphics[width=0.625\linewidth]{Figs/umbc_total_vs_atm_vs_GHG.pdf}
%  \end{center}
  \hspace{0.50in} Clouds/Atmos/GHG  \hspace{1.5in} Ind. Gases \\
  \begin{center}
    \dlandgraph{0.48}{Figs/umbc_total_vs_atm_vs_GHG.pdf}{Figs/umbc_total_vs_atm_vs_GHG2.pdf}
  \end{center}
 Total = Cloud/Atm/GHG where \newline
 \textcolor{blue}{Cloud=}Water+Ice, \textcolor{blue}{GHG=}\cd,\methane,\nitrous, \textcolor{blue}{Atm=}SurfTemp,T,\water,\ozone \newline
 Errorbars : TBD for next AIRS STM, cloud signal pretty small
\end{frame}

\begin{frame}
  \frametitle{Discussion}
  \begin{itemize}
    \item This is a validation of our spectral trend $\rightarrow$ geophysical rates
    \item Need to improve error estimates and cloud geophysical rates
    \item CO2 flux change contribution < -0.5 of the total flux change (ie reducing the OLR)
    \item WV  flux change contribution is typically very small except at -30 S (-0.5 of the total flux change (ie reducing the OLR))
          \textcolor{red}{and at equator where is it -2 of total flux change, leading to overall OLR reduction here}
    \item SurfTemp flux change contribution $\sim$ +0.5 of total flux change (ie increasing the OLR)
    \item AtmTemp  flux change contribution $\sim$ > +1.0 of total flux change (ie increasing the OLR)            
  \end{itemize}
  Tatm, Tsurf emit more OLR than what GHG (CO2/WV) are trapping \newline
  In Southern Ocean region, lowered Surf Temps causing net reduction in OLR
  \end{frame}

\begin{frame}
  \frametitle{Conclusions}
  \begin{itemize}
  \item Comparison of our trend retrievals to CERES OLR provides partial validation of our trends.
  \item RRTM allows us to dissect changes to OLR according to spectral region
  \item TwoSlab fluxes "not too bad"!
 % \item ERA TwoSlab cloud trends $\rightarrow$ RRTM : large change in cirrus fraction
  %       produces unrealistically large OLR changes
  \item UMBC Spectral Rates $\rightarrow$ Thermodynamic+Cloud rates \textcolor{red}{good agreement
        with OLR rates from CERES}. 
  \item Need to work more on errors \textcolor{red}{Note the rates are on order of 0.05 W/m2/yr while initial error estimate is about 5-10 times smaller}  
  \item To do list : try MRO clouds in RRTM ....
  \end{itemize}
\end{frame}
% ---------------------------------------------------------------------
% ---------------------------------------------------------------------

\end{document}

% ---------------------------------------------------------------------
% ---------------------------------------------------------------------
\section{Extra Slides : Flux Changes by RRTM bands : ERA clouds}
% ---------------------------------------------------------------------
% ---------------------------------------------------------------------
\begin{frame}
  \frametitle{Flux Changes by RRTM bands}
  RED means MORE flux (W/m2) in 2017/2018 \newline
  \hspace{0.50in} Clear Sky  \hspace{1.75in} All Sky \\
  \begin{center}
    \dlandgraph{0.48}{Figs/spectral_deltaolr_vs_lat_thermo_gases.png}{Figs/spectral_deltaolr_vs_lat_thermo_gases_clouds.png}
  \end{center}

  \begin{small}
    \begin{itemize}
    \item \textcolor{red}{fix all this, loooks wrong}
    \item LH panel, can see in 15 um band the change is red
    \item LH panel, can see more clr sky flux coming out in Southern Polar Regions in thermal window
    \item RH panel, more flux coming out in tropical window region in 2002
    \item RH panel, can see topical WV region had more flux emitted in 2002
    \end{itemize}
  \end{small}
\end{frame}


\end{document}
% ---------------------------------------------------------------------
% ---------------------------------------------------------------------

% ---------------------------------------------------------------------
% ---------------------------------------------------------------------
\section{Time series}
% ---------------------------------------------------------------------
% ---------------------------------------------------------------------
\begin{frame}
  \frametitle{Calculated Time Series (using ERA)}
  Monthly W/m2 from 2002/09 to 2017/08 \newline
  \hspace{0.50in} No Clouds  \hspace{1.75in} TwoSlab Clouds \\
  \begin{center}
    \dlandgraph{0.48}{Figs/olr_lat_time_nocloud.pdf}{Figs//olr_lat_time_cloud.pdf}
  \end{center}
\end{frame}
% ---------------------------------------------------------------------
%\begin{frame}
%  \frametitle{Greenhouse Effect}
%  Monthly W/m2 from 2002/09 to 2017/08 \newline
%  We know monthly averaged surface temperatures so we know power
%  emitted at surface, so we can compute GH effect (with clouds is
%  larger than without clouds) \newline 
%  \hspace{0.50in} GHG only \hspace{1.75in} GHG + CLouds \\
%  \begin{center}
%    \dlandgraph{0.48}{Figs/ghgeffect_delta_sb_olr_lat_time_nocloud.pdf}{Figs/ghgNcloudeffect_delta_sb_olr_lat_time.pdf}
%  \end{center}
%\end{frame}
% ---------------------------------------------------------------------
\begin{frame}
  \frametitle{Comparisons to CERES}
  CERES OLR - ERA TwoSlab OLR  (W/m2)
  \begin{center}
    \noindent\includegraphics[width=0.625\linewidth]{Figs/ceresVSghgNcloudeffect_lat_time.pdf}
  \end{center}
  So clearly ERA clouds $\rightarrow$ TwoSlab/Fluxes shows much larger changes in OLR than observed!
\end{frame}

%%%%%%%%%%%%%%%%%%%%%%%%%%%%%%%%%%%%%%%%%%%%%%%%%%%%%%%%%%%%%%%%%%%%%%%%
